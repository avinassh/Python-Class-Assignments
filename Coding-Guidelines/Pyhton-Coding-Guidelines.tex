% Created 2013-11-20 Wed 16:23
\documentclass[11pt]{article}
\usepackage[utf8]{inputenc}
\usepackage[T1]{fontenc}
\usepackage{fixltx2e}
\usepackage{graphicx}
\usepackage{longtable}
\usepackage{float}
\usepackage{wrapfig}
\usepackage{soul}
\usepackage{textcomp}
\usepackage{marvosym}
\usepackage{wasysym}
\usepackage{latexsym}
\usepackage{amssymb}
\usepackage{hyperref}
\tolerance=1000
\providecommand{\alert}[1]{\textbf{#1}}

\title{Python Coding Guidelines}
\author{Devi Prasad, Chandan}
\date{2013-11-20}
\hypersetup{
  pdfkeywords={},
  pdfsubject={},
  pdfcreator={Emacs Org-mode version 7.9.3f}}

\begin{document}

\maketitle

\setcounter{tocdepth}{3}
\tableofcontents
\vspace*{1cm}

\section{Purpose of the document}
\label{sec-1}


  This document provides the coding guidelines for Python programming language based on \href{http://www.python.org/dev/peps/pep-0008/}{http://www.python.org/dev/peps/pep-0008/}.
\section{Modification Log}
\label{sec-2}


\begin{center}
\begin{tabular}{rlll}
\hline
  Rev  &  Description    &  Author   &  Date                     \\
\hline
 0.01  &  Initial draft  &  Avinash  &  \textit{2013-11-20 Wed}  \\
\hline
\end{tabular}
\end{center}
\section{Code lay-out}
\label{sec-3}
\subsection{Indentation}
\label{sec-3-1}

\begin{itemize}
\item Use 4 spaces per indentation level (no tabs).
\end{itemize}
\subsection{Maximum Line Length}
\label{sec-3-2}

\begin{itemize}
\item Limit all lines to a maximum of 79 characters.
\end{itemize}
\subsection{Blank Lines}
\label{sec-3-3}

\begin{itemize}
\item Separate top-level function and class definitions with two blank lines.
\item Method definitions inside a class are separated by a single blank line.
\item Use blank lines in functions, sparingly, to indicate logical sections.
\end{itemize}
\subsection{Imports}
\label{sec-3-4}

\begin{itemize}
\item Imports should usually be on separate lines
\item Imports are always put at the top of the file, just after any module comments and docstrings, and before module globals and constants.
\item Imports should be grouped in the following order:
\begin{enumerate}
\item standard library imports
\item related third party imports
\item local application/library specific imports
\end{enumerate}
\end{itemize}
   You should put a blank line between each group of imports.

   Put any relevant \_\_all\_\_ specification after the imports.
\subsection{Whitespace in Expressions and Statements}
\label{sec-3-5}

\begin{itemize}
\item \href{http://www.python.org/dev/peps/pep-0008/#id17}{http://www.python.org/dev/peps/pep-0008/\#id17} (stick to all of the points)
\end{itemize}
\subsection{Comments}
\label{sec-3-6}

\begin{itemize}
\item State why and now how'
\end{itemize}
\subsection{Documentation Strings}
\label{sec-3-7}

\begin{itemize}
\item Write docstrings for all public modules, functions, classes, and methods. Docstrings are not necessary for non-public methods, but you should have a comment that describes what the method does. This comment should appear after the def line.
\end{itemize}
\section{Naming Conventions}
\label{sec-4}
\subsection{Overriding Principle}
\label{sec-4-1}

\begin{itemize}
\item Names that are visible to the user as public parts of the API should follow conventions that reflect usage rather than implementation.
\end{itemize}
\subsection{Descriptive: Naming Styles}
\label{sec-4-2}

\begin{itemize}
\item \href{http://www.python.org/dev/peps/pep-0008/#descriptive-naming-styles}{http://www.python.org/dev/peps/pep-0008/\#descriptive-naming-styles}
\end{itemize}
\subsection{Package and Module Names}
\label{sec-4-3}

\begin{itemize}
\item Modules should have short, all-lowercase names. Underscores can be used in the module name if it improves readability. Python packages should also have short, all-lowercase names, although the use of underscores is discouraged.
\end{itemize}
\subsection{Class Names}
\label{sec-4-4}

\begin{itemize}
\item Class names should normally use the CapWords convention.
\end{itemize}
\subsection{Exception Names}
\label{sec-4-5}

\begin{itemize}
\item Because exceptions should be classes, the class naming convention applies here. However, you should use the suffix ``Error'' on your exception names (if the exception actually is an error).
\end{itemize}
\subsection{Global Variable Names}
\label{sec-4-6}

\begin{itemize}
\item Modules that are designed for use via from M import * should use the \_\_all\_\_ mechanism to prevent exporting globals.
\end{itemize}
\subsection{Function Names}
\label{sec-4-7}

\begin{itemize}
\item Function names should be lowercase, with words separated by underscores as necessary to improve readability.
\end{itemize}
\subsection{Function and method arguments}
\label{sec-4-8}

\begin{itemize}
\item Always use self for the first argument to instance methods.
\item Always use cls for the first argument to class methods.
\end{itemize}
\subsection{Method Names and Instance Variables}
\label{sec-4-9}

\begin{itemize}
\item Use the function naming rules: lowercase with words separated by underscores as necessary to improve readability.
\item Use one leading underscore only for non-public methods and instance variables.
\end{itemize}
\subsection{Constants}
\label{sec-4-10}

\begin{itemize}
\item Constants are usually defined on a module level and written in all capital letters with underscores separating words. Examples include MAX$_{\mathrm{OVERFLOW}}$ and TOTAL.
\end{itemize}
\subsection{Designing for inheritance}
\label{sec-4-11}

\begin{itemize}
\item \href{http://www.python.org/dev/peps/pep-0008/#id38}{http://www.python.org/dev/peps/pep-0008/\#id38} (read the complete section)
\end{itemize}
\subsection{Public and internal interfaces}
\label{sec-4-12}

\begin{itemize}
\item Any backwards compatibility guarantees apply only to public interfaces. Accordingly, it is important that users be able to clearly distinguish between public and internal interfaces.
\item To better support introspection, modules should explicitly declare the names in their public API using the \_\_all\_\_ attribute. Setting \_\_all\_\_ to an empty list indicates that the module has no public API.
\item Even with \_\_all\_\_ set appropriately, internal interfaces (packages, modules, classes, functions, attributes or other names) should still be prefixed with a single leading underscore.
\end{itemize}
\section{Programming Recommendations}
\label{sec-5}

\begin{itemize}
\item \href{http://www.python.org/dev/peps/pep-0008/#id40}{http://www.python.org/dev/peps/pep-0008/\#id40}
\end{itemize}

\end{document}
